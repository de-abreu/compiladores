\documentclass[a4paper]{article}

% Package for formatting lists and printing mathematical formulas
\usepackage[brazil]{babel}
\usepackage[utf8]{inputenc}
\usepackage{amsmath}
\usepackage{amssymb}
\usepackage{biblatex}
\usepackage{csquotes}
% 1. Tell LaTeX the input encoding is UTF-8
\usepackage[utf8]{inputenc}

% 2. Use the T1 font encoding for better output of accented characters
\usepackage[T1]{fontenc}

% 3. Optional: Improve font appearance (optional, but nice)
\usepackage{forest}
\usepackage{lmodern} % Or times, etc.
\usepackage{tabularx}
\usepackage{ragged2e}

% Formatting commands
\setlength{\parindent}{0pt}
\setlength{\parskip}{1.5\baselineskip}
\newcommand{\addauthor}[2]{#1~(nUSP: #2)\\}
\newcommand{\exerciseList}[1]{
	\title{Lista de Exercícios #1: Resolução}
	\author{
		\addauthor{Guilherme de Abreu}{12543033}
		\addauthor{Hélio Cardoso}{10310227}
		\addauthor{Laura Camargos}{13692334}
		\addauthor{Sandy Dutra}{12544570}
		\addauthor{Theo dos Santos}{10691331}
	}
}


% Format exercises
\newcounter{count}
\setcounter{count}{0}
\renewcommand{\labelenumi}{\alph{enumi}.}
\newcommand{\exercise}[2]{%
  \stepcounter{count}%
  \noindent\textbf{Exercício \thecount:}

  #1
  
  \noindent\textbf{Resolução:}%
  
  #2 
  
}
\exerciseList{3}

\begin{document}
\maketitle

\exercise{
	Considere o seguinte programa escrito em LALG:

	program p1;\\
	var x: integer;\\
	begin\\
	\hspace*{2em} read(x);\\
	\hspace*{2em} x := x * 2;\\
	\hspace*{2em} write(x);\\
	end.

	Em um processo de compilação, qual seria a saída da

	\begin{enumerate}
		\item análise léxica?
		\item análise sintática?
		\item análise semântica?
	\end{enumerate}
} {
	% Respostas ao exercício 1
	\begin{enumerate}
		\item uma sequência de palavras (tokens) associadas a lexemas
		      reconhecidos pela linguagem. Ex.:

		      \texttt{<program, keyword> <p1, id>\\
			      <var, keyword> <x, id2> <:, symbol>\\
			      <integer, keyword><;, symbol>\\
			      <begin, keyword>\\
			      <read, keyword> <(, symbol> <x, id2> <), symbol>\\
			      <;, symbol>\\
			      <x, id2> <:=, operator> <x, id2> <*, operator> <2, number>\\
			      <;, symbol>\\
			      <write, keyword> <(, symbol> <x, id2> <), symbol>\\
			      <;, symbol>\\
			      <end., keyword> <., symbol>\\}

		\item uma árvore sintática abstrata (AST), uma estrutura de dados
		      que organiza os tokens hierarquicamente, segundo as normas
		      gramaticais da linguagem. Uma possível representação em gráfica
		      simplificada desta seria:

		      \begin{forest}
			      [PROGRAM
					      [\texttt{program p1;}]
					      [VAR\_DECL
						      [\texttt{x: integer;}]
					      ]
					      [BEGIN
							      [PROC\_CALL
								      [\texttt{read(x)}]
							      ]
							      [ASSIGNMENT
									      [\texttt{x :=}
									      ]
									      [EXPRESSION
											      [\texttt{x}]
											      [\texttt{*}]
											      [\texttt{2}]
									      ]
							      ]
							      [PROC\_CALL
								      [\texttt{write(x)}]
							      ]
					      ]
					      [\texttt{END.}]
			      ]
		      \end{forest}

		\item finalmente, a análise semântica é o estágio que verifica se o
		      código-fonte é significativo para além da sua sintaxe. Isto é,
		      para além de estar bem formatado ele se traduz em instruções
		      válidas. Isto inclui:

		      \begin{itemize}
			      \item Verificação de tipos
			      \item Verificação de escopo
			      \item Validação de chamadas de função/procedimento
			      \item A construção de uma tabela de símbolos que sumariza as
			            etapas anteriores
		      \end{itemize}

		      Um exemplo de tabela de símbolos para presente código seria:

		      \begin{tabularx}{\linewidth}{|l|X|}
			      \hline
			      \textbf{Identifier} & \textbf{Type} \\
			      \hline
			      x                   & integer       \\
			      \hline
		      \end{tabularx}

		      Não apresentando falhas nestas sucessivas etapas, o código-fonte é
		      habilitado a ser compilado


	\end{enumerate}
}

\exercise{
	Defina os seguintes correlatos ao processo de compilação:

	\begin{enumerate}
		\item Interpretador
		\item Processador de macro
		\item Editor de ligação
		\item Montador (Assembler)
		\item Pré-processador
		\item Editor/IDE
		\item Depurador
	\end{enumerate}
}  {
	% Respostas ao exercício 2
	\begin{enumerate}
		\item \textbf{Interpretador:} programa o qual executa um dado
		      código-fonte à partir de sua leitura, linha a linha, durante o
		      tempo de execução.
		\item \textbf{Processador de macro:} ferramenta que processa
		      substituições de texto em um dado código-fonte antes de sua
		      compilação interpretação. Isto é, este reconhece dadas invocações de
		      macro (como o \texttt{#define} na linguagem C) e a substituem por
		      definições a estas atribuídas.
		\item \textbf{Editor de ligação (Linker):} programa o qual combina uma
		      série de arquivos (denominados "objetos") gerados pelo compilador em
		      um único arquivo executável, ao resolver as referências entre estes.
		\item \textbf{Montador (Assembler):} programa o qual provê tradução
		      entre uma linguagem de programação de baixo nível denominada
		      "linguagem de montagem" e correspondentes instruções de máquina.
		      Como a relação entre instruções de montagem e instruções de máquina
		      se aproximam de uma correspondência de 1 para 1, tal linguagem
		      permite ao programador ganhos em controle e desempenho em troca de
		      perdas em termos de menor portabilidade e maior complexidade.
		\item \textbf{Pré-processador:} programa o qual opera transformações no
		      código-fonte antes deste ser processado pelo compilador. Este
		      abarca a inclusão de arquivos, a ativação do processador de macros,
		      a compilação condicional, dentre outras várias funcionalidades.
		\item \textbf{Editor/IDE:} programa o qual provê um conjunto de
		      ferramentas em uma interface unificada para permitir ao usuário um
		      ambiente de desenvolvimento eficiente para -dentre outras
		      funcionalidades - escrever, testar e depurar outros programas à
		      partir do código-fonte destes.
		\item \textbf{Depurador:} programa o qual permite o teste, análise e
		      depuração de um dado programa alvo por meio do controle de execução,
		      inspeção de variáveis e rastreio de erros no código-fonte deste.
		      Ou seja, depuradores são ferramentas que  auxiliam programadores
		      a encontrar e reparar falhas em dados programas.
	\end{enumerate}
}

\exercise{
	Quais as principais similaridades e diferenças entre interpretadores e
	compiladores? Qual método é mais vantajoso?
} {
	São similaridades entre compiladores e interpretadores:

	\begin{itemize}
		\item \textbf{Propósito:} tanto compiladores quanto interpretadores são
		      ferramentas pelas quais um dado código-fonte pode ser traduzido a
		      uma forma que pode ser executada por um computador.

		\item \textbf{Detecção de erro:} Ambos realizam tarefas de análise
		      léxica, sintática e semântica para assegurar que o código-fonte é
		      válido. E podem orientar desenvolvedores a identificar e corrigir
		      falhas antes da execução deste.
	\end{itemize}

	Enquanto as principais divergências entre interpretadores e compiladores
	encontram-se descritas na tabela \ref{tab:comparacaoInterpreterCompiler}.

	Em suma, as características de linguagens interpretadas as tornam propícias
	de serem utilizadas em ambientes de desenvolvimento dinâmicos como, por
	exemplo, na prototipagem, na análise de dados, ou na escrita de scripts. Ou
	seja, situações as quais o custo adicional de sucessivos processos de
	compilação não se justifica. Enquanto, por outro lado linguagens compiladas
	são preferíveis quando é crítico o melhor desempenho e eficiência dos
	programas gerados.

	\begin{table}[htbp]
		\centering
		\caption{Principais divergências entre interpretadores e compiladores}
		\label{tab:comparacaoInterpreterCompiler}
		\begin{tabularx}{\linewidth}{ | l | >{\RaggedRight\arraybackslash}X | >{\RaggedRight\arraybackslash}X | }
			\hline
			\textbf{Aspecto}                                              & \textbf{Interpretador} & \textbf{Compilador} \\
			\hline
			Execução                                                      &
			Executa o código-fonte traduzido em linguagem de máquina, linha a
			linha, em tempo de execução                                   &
			Executa uma tradução do código-fonte feita previamente, em sua
			inteiridade.                                                                                                 \\
			\hline
			Saída                                                         &
			Não gera arquivo executável, o código é executado diretamente &
			Produz arquivo executável.                                                                                   \\
			\hline
			Tempo de execução                                             &
			Maior, pois o código é traduzido e executado simultaneamente. &
			Menor, a execução é feita diretamente.                                                                       \\
			\hline
			Consumo de memória                                            &
			Maior, pois o interpretador precisa ser carregado a memória durante
			execução.                                                     &
			Menor, o código é executado de forma independente.                                                           \\
			\hline
			Tratamento de erros                                           &
			Erros são todos detectados e relatados durante execução       &
			Alguns erros são detectados e relatados durante compilação                                                   \\
			\hline
			Portabilidade                                                 &
			Maior, dado código-fonte pode ser executado em qualquer plataforma
			alvo que possua um interpretador.                             &
			Menor, o código-fonte necessita ser compilado para gerar um arquivo
			executável cada plataforma alvo.                                                                             \\
			\hline
			Depuração                                                     &
			Mais ágil pois se tem uma resposta imediata a mudanças feitas no
			código.                                                       &
			Mais lenta pois cada modificação exige recompilação.                                                         \\
			\hline
		\end{tabularx}
	\end{table}

}

\newpage

\exercise {
	Quais as características de uma linguagem que determinam que ela deve
	ser compilada ou interpretada? Esta questão refere-se à linguagem em si,
	independentemente do uso que é feito dela.
} {
	Não há elementos na linguagem que conclusivamente apontem se esta a de ser
	interpretada ou compilada, tido que estas são características da
	implementação da linguagem Não obstante, há características que fortemente
	sugerem a linguagem ser utilizada para um fim e não outro. Por exemplo,
	linguagens dinamicamente tipadas tendem a ser interpretadas, pois esta é uma
	facilidade que pode ser resolvida em tempo de execução.
}

\end{document}

