\documentclass[a4paper]{article}

% Package for formatting lists and printing mathematical formulas
\usepackage[brazil]{babel}
\usepackage[utf8]{inputenc}
\usepackage{amsmath}
\usepackage{amssymb}
\usepackage{biblatex}
\usepackage{csquotes}
% 1. Tell LaTeX the input encoding is UTF-8
\usepackage[utf8]{inputenc}

% 2. Use the T1 font encoding for better output of accented characters
\usepackage[T1]{fontenc}

% 3. Optional: Improve font appearance (optional, but nice)
\usepackage{forest}
\usepackage{lmodern} % Or times, etc.
\usepackage{tabularx}
\usepackage{ragged2e}

% Formatting commands
\setlength{\parindent}{0pt}
\setlength{\parskip}{1.5\baselineskip}
\newcommand{\addauthor}[2]{#1~(nUSP: #2)\\}
\newcommand{\exerciseList}[1]{
	\title{Lista de Exercícios #1: Resolução}
	\author{
		\addauthor{Guilherme de Abreu}{12543033}
		\addauthor{Hélio Cardoso}{10310227}
		\addauthor{Laura Camargos}{13692334}
		\addauthor{Sandy Dutra}{12544570}
		\addauthor{Theo dos Santos}{10691331}
	}
}


% Format exercises
\newcounter{count}
\setcounter{count}{0}
\renewcommand{\labelenumi}{\alph{enumi}.}
\newcommand{\exercise}[2]{%
    \stepcounter{count}%
    \noindent\textbf{Exercício \thecount:} #1

    \noindent\textbf{Resolução:}%
    
    #2

    \pagebreak
}

\exerciseList{7}

\begin{document}
\maketitle
\exercise{
    Criar uma tabela sintática para a gramática abaixo.

    $S \to AS' \\
    S' \to S \mid \lambda \\
    A \to a \mid b \mid \lambda$
}{
    Iniciamos pela construção da Tabela de Primeiros e Seguidores:

    \begin{table}[htbp]
        \centering
        \begin{tabularx}{\linewidth}{ | l | >{\RaggedRight\arraybackslash}X | >{\RaggedRight\arraybackslash}X | }
            \hline
            \textbf{Regra}                        & \textbf{Primeiro}  & \textbf{Seguidor} \\
            \hline
            $S \to AS'$                   & $a, b, +, \lambda$ & $\lambda$         \\
            \hline
            $S' \to S \mid \lambda$       & $+, \lambda$       & $\lambda$         \\
            \hline
            $A \to a \mid b \mid \lambda$ & $a, b, \lambda$    & $+, \lambda$      \\
            \hline
        \end{tabularx}
    \end{table}

    Com base nesta, determinamos as regras de transição para obtenção dos
    símbolos terminais a partir dos terminais, que orienta a análise preditiva
    não recursiva:

    \begin{table}[htbp]
        \centering
        \begin{tabularx}{\linewidth}{ | l | >{\RaggedRight\arraybackslash}X |
                        >{\RaggedRight\arraybackslash}X | >{\RaggedRight\arraybackslash}X |
                        >{\RaggedRight\arraybackslash}X | }
            \hline
                 & \textbf{a}          & \textbf{b}          & \textbf{+}          &
            $\lambda$                                                                               \\
            \hline
            $S$  & $S \to AS'$ & $S \to AS'$ & $S \to AS'$ & $S \to AS'$ \\
            \hline
            $S'$ & $S' \to +S$ &                     & $S' \to +S$ & $S' \to \lambda$ \\
            \hline
            $A$ & $A \to a$ & $A \to b$                    & &
            $A \to \lambda$ \\
            \hline
        \end{tabularx}
    \end{table}

}

\exercise{
    Construa a tabela sintática para a gramática abaixo e reconheça a cadeia
    $id+id\cdot id$ utilizando análise sintática preditiva não recursiva. A cada
    passo, mostre o estado da pilha, da cadeia de entrada e a regra utilizada na
    derivação. Mostre a implementação deste analisador sintático.

    $E \to TE' \\
    E' \to +TE' \mid \lambda \\
    T \to FT' \\
    T' \to \cdot FT' \mid \lambda \\
    F \to \mid id$
}{
    Seguindo os passos vistos no exercício anterior, temos:

    \begin{table}[htbp]
        \centering
        \begin{tabularx}{\linewidth}{ | l | >{\RaggedRight\arraybackslash}X | >{\RaggedRight\arraybackslash}X | }
            \hline
            \textbf{Regra}                        & \textbf{Primeiro}  & \textbf{Seguidor} \\
            \hline
            $E \to TE'$                   & $(, id$ & $), \lambda$         \\
            \hline
            $E' \to +TE' \mid \lambda$       & $+, \lambda$       & $),
            \lambda$         \\
            \hline
            $T \to FT'$ & $(, id$    & $+, \lambda$      \\
            \hline
            $T' \to \cdot FT' \mid \lambda$ & $\cdot, \lambda$    & $+, \lambda$      \\
            \hline
            $F \to (E) \mid id$ & $(, id$ & $\cdot, \lambda$ \\
            \hline
        \end{tabularx}
    \end{table}

    E a seguinte tabela sintática:

    \begin{table}[htbp]
        \centering
        \begin{tabularx}{\linewidth}{ | l | >{\RaggedRight\arraybackslash}X |
                        >{\RaggedRight\arraybackslash}X | >{\RaggedRight\arraybackslash}X |
                        >{\RaggedRight\arraybackslash}X |
                    >{\RaggedRight\arraybackslash}X |
                >{\RaggedRight\arraybackslash}X |}
            \hline
                 & \textbf{(}          & \textbf{)}          & \textbf{+}          &
            $\mathbf \cdot$
            	 & \textbf{id} & $\mathbf \lambda$ \\
            \hline
           		$E$ & $\to TE'$ & & & & $ \to TE'$ & \\
           	\hline
           		$E'$ & & $\to \lambda$ & $\to +TE'$ & & & $\to
           		\lambda$ \\
           	\hline
           		$T$ & $\to FT'$ & & $\to FT'$ & & $\to FT'$ & \\
           	\hline
           		$T'$ & & & $\to \lambda$ & $\to FT'$ & & $\to \lambda$ \\
           	\hline
           		$F$ & $\to (E)$ & & & & $\to id$ & \\
           	\hline
        \end{tabularx}
    \end{table}

    O que nos leva ao seguinte processo de análise sintática não recursivo:

        \begin{table}[htbp]
        \centering
        \begin{tabularx}{\linewidth}{ | l | >{\RaggedRight\arraybackslash}X |
                        >{\RaggedRight\arraybackslash}X |}
            \hline
            \textbf{Stack} & \textbf{String} & \textbf{Regra} \\
            \hline
            $\lambda E$ & $id + id \cdot id \lambda$ & $E \to TE'$ \\
           	\hline
            $\lambda E'T$ & $id + id \cdot id \lambda$ & $T \to FT'$ \\
            \hline
            $\lambda E'T'F$ & $id + id \cdot id \lambda$ & $F \to id$ \\
            \hline
            $\lambda E'T'$ & $+ id \cdot id \lambda$ & $T' \to \lambda$ \\
            \hline
            $\lambda E'$ & $+ id \cdot id \lambda$ & $E' \to +TE'$ \\
            \hline
            $\lambda E'T$ & $id \cdot id \lambda$ & $T \to FT'$ \\
            \hline
            $\lambda E'T'F$ & $id \cdot id \lambda$ & $F \to id$ \\
            \hline
            $\lambda E'T'$ & $\cdot id \lambda$ & $T' \to \cdot FT'$ \\
            \hline
            $\lambda E'T'F$ & $id \lambda$ & $F \to id$ \\
            \hline
            $\lambda E'T'$ & $\lambda$ & $T' \to \lambda$ \\
            \hline
            $\lambda E'$ & $\lambda$ & $E' \to \lambda$ \\
            \hline
            $\lambda$ & $\lambda$ & Sucesso \\
            \hline
        \end{tabularx}
    \end{table}

}

\end{document}
