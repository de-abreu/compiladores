\documentclass[a4paper]{article}

% Package for formatting lists and printing mathematical formulas
\usepackage[brazil]{babel}
\usepackage[utf8]{inputenc}
\usepackage{amsmath}
\usepackage{amssymb}
\usepackage{biblatex}
\usepackage{csquotes}

% Formatting commands
\setlength{\parindent}{0pt}
\setlength{\parskip}{1.5\baselineskip}
\newcommand{\addauthor}[2]{#1~(nUSP: #2)\\}
\newcommand{\exerciseList}[1]{
	\title{Lista de Exercícios #1: Resolução}
	\author{
		\addauthor{Guilherme de Abreu}{12543033}
		\addauthor{Hélio Cardoso}{10310227}
		\addauthor{Laura Camargos}{13692334}
		\addauthor{Sandy Dutra}{12544570}
		\addauthor{Theo dos Santos}{10691331}
	}
}


% Format exercises
\newcounter{count}
\setcounter{count}{0}
\renewcommand{\labelenumi}{\alph{enumi}.}
\newcommand{\exercise}[2]{%
	\stepcounter{count}%
	\noindent\textbf{Exercício \thecount:}

	\nopagebreak

	#1

	\noindent\textbf{Resolução:}%

	\nopagebreak
	
	#2

}
\exerciseList{5}

\begin{document}
\maketitle

Todos os seguintes exercícios fazem uso das convenções \textbf{Regex} para
indicação da captura de caracteres. Sendo que, para um dado estado, a captura de
quaisquer caracteres os quais não estejam contemplados noutras regras de transição
é indicada por ".". Senão pelo primeiro exercício, onde a necessidade de
exemplificação da captura de erros se faz presente, todos os demais exercícios
fazem uso de autômatos finitos \textit{não determinísticos}. Não obstante, estes
poderiam ser facilmente convertidos em autômatos finitos adicionando a todos os
estados, senão aqueles de aceite, um regra de transição "." (ou "$", no caso do
exercício 3) que leva a um estado de captura (\textit e), como o seguinte:

\begin{center}
	\begin{tikzpicture}
		\node (e) [state, initial, accepting, initial text=.] {$e$};
		\node (label) [right=0.2cm of e]
		{\parbox{7cm}{\texttt{print("unknown token (\%s)", string)}
				\\\texttt{move\_back()}}};


		\path [-stealth, thick]
		(e) edge [loop above]
		node {$[\hat\ \ \textbackslash t \textbackslash n]$} ();
	\end{tikzpicture}
\end{center}

\pagebreak

\exercise{Adiciona o tratamento de erros no autômato de reconhecimento de
	identificadores. Como os seguintes erros seriam reconhecidos?

	@minha\_variavel\\
	minha@\_variavel\\
	minha\_variavel@}{

	\begin{center}
		\begin{tikzpicture} [node distance= 3cm, on grid, auto]
			% States
			\node (q0) [state, initial, initial text =]{$q_0$};
			\node (q1) [state, right= of q0]{$q_1$};
			\node (q2) [state, accepting, right= of q1]{$q_2$};
			\node (e) [state, accepting, below= of q1]{$e$};
			\node (error) [right=4.5cm of e]
			{\parbox{7cm}{\texttt{print("unknown token (\%s)", string)}
					\\\texttt{move\_back()}}};
			\node (return) [right=4.5cm of q2]
			{\parbox{7cm}{\texttt{return(string)}
					\\\texttt{move\_back()}}};



			% Transitions
			\path [-stealth, thick]
			(q0) edge [bend right] node [left] {.} (e)
			(q0) edge node [above] {[a-zA-Z]} (q1)
			(q1) edge [loop above] node {[a-zA-Z0-9\_]} ()
			(q1) edge node [left] {.} (e)
			(e) edge [loop below]
			node {$[\hat\ \ \textbackslash t \textbackslash n]$} ()
			(q1) edge node [above] {[ \textbackslash t\textbackslash n]} (q2);
		\end{tikzpicture}
	\end{center}
}

\pagebreak

\exercise{Crie um autômato para realizar a análise léxica de números em ponto
	flutuante. Exemplos de números aceitos e não aceitos:

	\nopagebreak

	Aceito: +1.23E+12\\
	Aceito: 1.23E12\\
	Aceito: -1.2492E-1\\
	Não aceito: 6.02E23.1}{
	\begin{center}
		\begin{tikzpicture}[node distance=2.5cm, on grid, auto]
			% States
			\node (q0) [state, initial, initial text =] {$q_0$};
			\node (q1) [state, above right=of q0] {$q_1$};
			\node (q2) [state, below right=of q1] {$q_2$};
			\node (q3) [state, right=of q2] {$q_3$};
			\node (q4) [state, accepting, above right=of q3] {$q_4$};
			\node (q5) [state, right=of q3] {$q_5$};
			\node (q6) [state, above right=of q5] {$q_6$};
			\node (q7) [state, right=of q5] {$q_7$};
			\node (return) [above right=1cm of q4]
			{\parbox{7cm}{\texttt{return(string)}\\\texttt{move\_back()}}};

			% Transitions
			\path[-stealth, thick]
			(q0) edge [bend left] node [left=0.2cm] {[+-]} (q1)
			(q0) edge node [above] {[0-9]} (q2)
			(q1) edge [bend left] node [right=0.2cm] {[0-9]} (q2)
			(q2) edge [loop below] node {[0-9]} ()
			(q2) edge node [above] {\textbackslash .} (q3)
			(q3) edge [loop below] node {[0-9]} ()
			(q3) edge [bend left] node [left=0.2cm]
				{[ \textbackslash t\textbackslash n]} (q4)
			(q3) edge node [above] {E} (q5)
			(q5) edge [bend right] node {[0-9]} (q6)
			(q5) edge [bend right] node [below] {[+-]} (q7)
			(q6) edge [bend right] node [above]
				{[ \textbackslash t\textbackslash n]} (q4)
			(q7) edge [bend right] node [right] {[0-9]} (q6)
			(q6) edge [loop above] node [right=0.1cm] {[0-9]} ();
		\end{tikzpicture}
	\end{center}
}

\pagebreak

\exercise{Construir autômato para consumir comentários:

	\nopagebreak

	\{essa função seve para...\}\\
	/*essa função serve para...*/
}{
	\begin{center}
		\begin{tikzpicture}[node distance=2.5cm, on grid, auto]
			% States
			\node (q0) [state, initial, initial text =] {$q_0$};
			\node (q1) [state, above right=of q0] {$q_1$};
			\node (q2) [state, right=of q1] {$q_2$};
			\node (q3) [state, above=of q2] {$q_3$};
			\node (q4) [state, right=of q2] {$q_4$};
			\node (q5) [state, accepting, below=of q4] {$q_5$};
			\node (q6) [state, below right=of q0] {$q_6$};
			\node (q7) [state, below right=of q6] {$q_7$};
			\node (return) [right=of q5]
			{{\texttt{return(string)}}};

			% Transitions
			\path[-stealth, thick]
			(q0) edge [bend left] node [left=0.2cm] {\textbackslash /} (q1)
			(q1) edge node [below] {\textbackslash *} (q2)
			(q2) edge [loop below] node {.}()
			(q2) edge [bend left] node [left=0.2cm]
				{\textbackslash \textbackslash} (q3)
			(q3) edge [loop above] node {\textbackslash \textbackslash} ()
			(q3) edge [bend left] node [right=0.2cm] {.} (q2)
			(q2) edge [bend left] node [above right=0.2cm]
				{\textbackslash *} (q4)
			(q4) edge [bend left] node [below] {.} (q2)
			(q4) edge [loop right] node {\textbackslash *} ()
			(q4) edge node {\textbackslash /} (q5)

			(q0) edge [bend right] node {\{} (q6)
			(q6) edge [loop above] node {.} ()
			(q6) edge [bend right] node {\textbackslash \textbackslash} (q7)
			(q7) edge [loop below] node {\textbackslash \textbackslash} ()
			(q7) edge [bend right] node [above right] {.} (q6)
			(q6) edge node [above left] {\}} (q5);
		\end{tikzpicture}
	\end{center}
}

\pagebreak

\exercise{Construa um autômato não não determinístico para identificar a
	seguinte linguagem: a(((b|a)\*c)d)\*kd\*a}{
	\begin{center}
		\begin{tikzpicture}[node distance=2cm, on grid, auto]
			% States
			\node (q0) [state, initial, initial text=] {$q_0$};
			\node (q1) [state, right=of q0] {$q_1$};
			\node (q2) [state, above right=of q1] {$q_2$};
			\node (q3) [state, below right=of q2] {$q_3$};
			\node (q4) [state, right=of q3] {$q_4$};
			\node (q5) [state, right=of q4] {$q_5$};
			\node (q6) [state, accepting, below=of q5] {$q_6$};
			\node (return) [right=of q6]
			{{\texttt{return(string)}}};

			% Transitions
			\path[-stealth, thick]
			(q0) edge node {a} (q1)
			(q1) edge [bend left] node {[a|b]} (q2)
			(q1) edge node {c} (q3)
			(q1) edge [bend right] node [below] {k} (q5)
			(q2) edge [bend right] node {c} (q3)
			(q3) edge [bend left] node {d} (q4)
			(q4) edge [bend left] node {c} (q3)
			(q4) edge [bend right] node [above right] {[a|b]} (q2)
			(q4) edge [bend left] node {k} (q5)
			(q5) edge node {a} (q6);
		\end{tikzpicture}
	\end{center}
}
\end{document}
