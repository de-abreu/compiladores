\documentclass[a4paper]{article}

% Package for formatting lists and printing mathematical formulas
\usepackage[brazil]{babel}
\usepackage[utf8]{inputenc}
\usepackage{amsmath}
\usepackage{amssymb}
\usepackage{biblatex}
\usepackage{csquotes}
% 1. Tell LaTeX the input encoding is UTF-8
\usepackage[utf8]{inputenc}

% 2. Use the T1 font encoding for better output of accented characters
\usepackage[T1]{fontenc}

% 3. Optional: Improve font appearance (optional, but nice)
\usepackage{forest}
\usepackage{lmodern} % Or times, etc.
\usepackage{tabularx}
\usepackage{ragged2e}

% Formatting commands
\setlength{\parindent}{0pt}
\setlength{\parskip}{1.5\baselineskip}
\newcommand{\addauthor}[2]{#1~(nUSP: #2)\\}
\newcommand{\exerciseList}[1]{
	\title{Lista de Exercícios #1: Resolução}
	\author{
		\addauthor{Guilherme de Abreu}{12543033}
		\addauthor{Hélio Cardoso}{10310227}
		\addauthor{Laura Camargos}{13692334}
		\addauthor{Sandy Dutra}{12544570}
		\addauthor{Theo dos Santos}{10691331}
	}
}


% Format exercises
\newcounter{count}
\setcounter{count}{0}
\renewcommand{\labelenumi}{\alph{enumi}.}
\newcommand{\exercise}[2]{%
  \stepcounter{count}%
  \noindent\textbf{Exercício \thecount:}

  #1
  
  \vspace{\baselineskip}%
  \noindent\textbf{Resolução:}%
  \vspace{\baselineskip}%
  
  #2 
  
  \vspace{2\baselineskip}%
}
\exerciseList{3}

\begin{document}
\maketitle

\exercise{
	Considere o seguinte programa escrito em LALG:

	program p1;\\
	var x: integer;\\
	begin\\
	\hspace*{2em} read(x);\\
	\hspace*{2em} x := x * 2;\\
	\hspace*{2em} write(x);\\
	end.

	Em um processo de compilação, qual seria a saída da

	\begin{enumerate}
		\item análise léxica?
		\item análise sintática?
		\item análise semântica?
	\end{enumerate}
} {
	% Respostas ao exercício 1
	\begin{enumerate}
		\item
		\item
		\item
	\end{enumerate}
}

\exercise{
	Defina os seguintes correlatos ao processo de compilação:

	\begin{enumerate}
		\item Interpretador
		\item Processador de macro
		\item Editor de ligação
		\item Montador (Assembler)
		\item Pré-processador
		\item Editor/IDE
		\item Depurador
	\end{enumerate}
}  {
	% Respostas ao exercício 2
	\begin{enumerate}
		\item
		\item
		\item
		\item
		\item
		\item
		\item
	\end{enumerate}
}

\exercise{
	Quais as principais similaridades e diferenças entre interpretadores e
	compiladores? Qual método é mais vantajoso?
} {
	% Respostas ao exercício 3
}

\exercise {
	Quais as características de uma linguagem que determinam que ela deve
	ser compilada ou interpretada? Esta questão refere-se à linguagem em si,
	independentemente do uso que é feito dela.
} {
	% Respostas ao exercício 4
}

\end{document}
